\documentclass[10pt, a4paper]{extarticle}
\usepackage{physics}
\usepackage{mathtools}
\usepackage{hyperref}
\usepackage{amssymb}
\usepackage{mathrsfs}
\usepackage[margin=0.75in]{geometry}
\usepackage{anyfontsize}
\usepackage{amsthm}
\usepackage{framed}

\theoremstyle{definition}
\newtheorem{thm}{Theorem}
\newtheorem{lem}{Lemma}[thm]
\newtheorem{cor}{Corollary}[thm]
\newtheorem{defn}{Definition}
\newtheorem{eg}{Example}
\newtheorem*{note*}{Note}

\begin{document}
\begin{center}
	\fontsize{25}{60}\selectfont Statistical Mechanics \\
	\large Based on lectures by Dr. Suvankar Dutta\\
	Notes taken by Rwik Dutta
\end{center}
\hrule
\begin{center}
	These notes are not endorsed by the lecturers, and I have modified them (often
	significantly) after lectures. They are nowhere near accurate representations of what
	was actually lectured, and in particular, all errors are almost surely mine.\footnote[1]{This is how Dexter Chua describes his lecture notes from Cambridge. I could not have described mine in any better way.}
\end{center}
\tableofcontents

\newpage

\section{Ensemble}
\begin{defn}[Macroscopic variables]
	The variables of a system that can be controlled externally or determined experimentally. They are also called thermodynamic variables.

	For e.g., volume, pressure, temperature, etc.
\end{defn}
\begin{defn}[Microscopic variables]
	These parameters are not under any control.

	For e.g., the coordinates of a molecule of a gaseuos system.
\end{defn}
\begin{defn}[Ensemble]
	Set of identical copies of the same system which are same at the macroscopic level but different at the microscopic level.
\end{defn}
\subsection{Phase Space}
In the case of 1D motion of a single particle, the phase space is represented by the $p,q$--plane, where $p,q$ represent the momentum and position of the paticle, respectively. The collection of points $(q,p)$ that are possible classical states of the particle is called the phase space trajectory(or surface) of the particle. This trajectory can be found by
\[E=\mathcal{H}(q,p)=\frac{p^2}{2m}+V(q)\]
where $E$ represents the classically allowed energy states of the system.
\begin{eg}[Classical Harmonic Oscillator]
	In this case $E$ is a constant of motion and $V(q)=-\frac{1}{2}m\omega^2q^2$. So we have
	\[\frac{p^2}{2m}-\frac{1}{2}m\omega^2q^2=E\]
	This is an ellipse in the $p,q$--plane. The direction of the trajectory depends on the boundary conditions.
\end{eg}
\begin{framed}
	Thus, the phase space trajectory tells us about the possible states of the particle(or, system) and how one state evolves into another.
\end{framed}
\hfill\\
Now, we extend this idea to $N$ particles in 3 dimensions. This time instead of just two quantities $p,q$, we require the 3 position and 3 momentum coordinates of each of the $N$ particles. Provided there are no additional constrains, in general, we require a $6N$ dimensional phase space to describe the states of this system. Thus, $E=\mathcal{H}$ represents a $6N-1$ dimensional hypersurface in this phase space.
\begin{note*}
	$\mathcal{H}$ is a function of all the $6N$ positions and momenta.
\end{note*}
\begin{note*}
	If $E$ is piecewise continuous, we get $6N$ dimensional regions in the phase space.
\end{note*}

\end{document}
