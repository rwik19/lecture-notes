\documentclass[10pt, a4paper]{extarticle}
\usepackage{physics}
\usepackage{mathtools}
\usepackage{hyperref}
\usepackage{amssymb}
\usepackage{mathrsfs}
\usepackage[margin=0.75in]{geometry}
\usepackage{anyfontsize}
\usepackage{amsthm}

\theoremstyle{definition}
\newtheorem{thm}{Theorem}
\newtheorem{lem}{Lemma}[thm]
\newtheorem{cor}{Corollary}[thm]
\newtheorem{defn}{Definition}
\newtheorem{eg}{Example}
\newtheorem{notation}{Notation}

\begin{document}
\begin{center}
	\fontsize{25}{60}\selectfont Statistical Mechanics \\
	\large Based on lectures by Dr. Suvankar Dutta\\
	Notes taken by Rwik Dutta
\end{center}
\hrule
\begin{center}
	These notes are not endorsed by the lecturers, and I have modified them (often
	significantly) after lectures. They are nowhere near accurate representations of what
	was actually lectured, and in particular, all errors are almost surely mine.\footnote[1]{This is how Dexter Chua describes his lecture notes from Cambridge. I could not have described mine in any better way.}
\end{center}
\tableofcontents

\newpage

\section{Mathematical Background}
\subsection{Dirac Delta Function}
\begin{defn}[Heaviside Step Function]
	\[\Theta(x)=\begin{cases}
			1, & x>0 \\
			0, & x<0
		\end{cases}\]
\end{defn}

\begin{defn}[Dirac Delta Function]
	The set of ``functions" $\delta_\epsilon:\mathbb{R}\to\mathbb{R}$ that satisfy
	\[\lim_{\epsilon\to 0^+}\delta_\epsilon(x)=\begin{cases}
		+\infty,&x=0\\
		0,&x\neq 0
	\end{cases}\]
	\[\lim_{\epsilon\to0^+}\int_{-\infty}^{\infty}\delta_\epsilon(x)\ dx=1\]
\end{defn}

\begin{eg}
	The Gaussian \[\delta_\epsilon(x)=\frac{1}{\epsilon\sqrt{\pi}}e^{-\frac{x^2}{\epsilon^2}}\] is a Dirac delta function.
\end{eg}

\begin{notation}
	\[\int_{-\infty}^{\infty}\delta(x)\ dx\equiv \lim_{\epsilon\to0^+}\int_{-\infty}^{\infty}\delta_\epsilon(x)\ dx\]
	Any integral involving the Dirac delta function must be interpreted as the corresponding integral of $\delta_\epsilon$ as $\epsilon\to 0^+$.
\end{notation}

\begin{thm}[Properties of the delta function]
	\hfill
	\begin{enumerate}
		\item $\delta(-x)=\delta(x)$
		\item $\delta(ax)=\frac{1}{|a|}\delta(x),\ a\neq 0$
		\item $\int_{-\infty}^{\infty}\delta(x-c)f(x)\ dx=f(c)$, for any function $f:\mathbb{R}\to\mathbb{R}$.
		\item $\delta(f(x))=\sum_i\frac{1}{|f'(x_i)|}\delta(x-x_i)$ if $f(x_i)=0$
\end{enumerate}
\end{thm}

\subsection{Gaussian Integral}
\begin{thm}
	\[\int_{-\infty}^{\infty}e^{-x^2}\ dx=\sqrt{\pi}\]
\end{thm}

\begin{thm}[Important Results of Gaussian integral]
	\hfill
	\begin{enumerate}
		\item $\int_{-\infty}^{\infty}e^{-ax^2+bx}\ dx=e^{\frac{b^2}{4a}}\sqrt{\frac{\pi}{a}}$
\end{enumerate}
\end{thm}

\begin{thm}
	In general,\[\int_0^\infty x^n e^{-ax^2}\ dx\propto a^{-\frac{n+1}{2}}\]
	and in particular,
	\[\int_0^\infty x^n e^{-ax^2}\ dx=\begin{cases}
		\frac{(n-1)(n-3)\cdots 3\cdot 1}{2^{\frac{n}{2}+1}a^{\frac{n}{2}}}\sqrt{\frac{\pi}{a}},&n\text{ even}\\
		\frac{[\frac{1}{2}(n-1)]!}{2a^{\frac{n+1}{2}}},&n\text{ odd}
	\end{cases}\]
\end{thm}

\section{Ensemble}
\begin{defn}[Macroscopic variables]
	The variables of a system that can be controlled externally or determined experimentally. They are also called thermodynamic variables.

	For e.g., volume, pressure, temperature, etc.
\end{defn}
\begin{defn}[Microscopic variables]
	These parameters are not under any control.

	For e.g., the coordinates of a molecule of a gaseuos system.
\end{defn}
\begin{defn}[Ensemble]
	Set of identical copies of the same system which are same at the macroscopic level but different at the microscopic level.
\end{defn}
\end{document}
