\documentclass[10pt, a4paper]{extarticle}
\usepackage{physics}
\usepackage{mathtools}
\usepackage{hyperref}
\usepackage{amssymb}
\usepackage{mathrsfs}
\usepackage[margin=0.75in]{geometry}
\usepackage{anyfontsize}
\usepackage{amsthm}

\theoremstyle{definition}
\newtheorem{thm}{Theorem}
\newtheorem{lem}{Lemma}[thm]
\newtheorem{cor}{Corollary}[thm]
\newtheorem{defn}{Definition}
\newtheorem{eg}{Example}

\begin{document}
\begin{center}
	\fontsize{25}{60}\selectfont Theory of Computation \\
	\large Based on lectures by Dr. Arpit Sharma\\
	Notes taken by Rwik Dutta
\end{center}
\hrule
\begin{center}
	These notes are not endorsed by the lecturers, and I have modified them (often
	significantly) after lectures. They are nowhere near accurate representations of what
	was actually lectured, and in particular, all errors are almost surely mine.\footnote[1]{This is how Dexter Chua describes his lecture notes from Cambridge. I could not have described mine in any better way.}
\end{center}
\tableofcontents

\newpage

\section{Finite Automata}
\begin{defn}[Deterministic Finite Automaton]
	A collection $(Q,\Sigma,\delta,q_0,F)$ such that
	\begin{enumerate}
		\item $Q$ is a finite set of \emph{states}.
		\item $\Sigma$ is a finite \emph{alphabet}.
		\item $\delta:Q\times\Sigma\to Q$ is the \emph{transition function}.
		\item $q_0\in Q$ is the start state.
		\item $F\subseteq Q$ is the set of \emph{accepting states}.
\end{enumerate}
\end{defn}
We will use automata to solve set membership problems, i.e., given a finite alphabet $\Sigma$ and a finite language $L\subset \Sigma^*$, we need to find whether a given string $x\in L$.

\begin{defn}[Language of Automaton]
	$L(M)$ of an automaton $M$ is the set of all strings $x$ that are accepted by the automaton.
\end{defn}

\begin{defn}[Extended transition function]
	Let $M(Q,\Sigma,\delta,q_0,F)$ be a finite automaton. The extended transition function is given by
	\begin{align*}
		\delta^*&:Q\to\Sigma^*\\
		\delta^*(q,\epsilon)&=q\\
		\delta^*(q,xa)&=\delta(\delta^*(q,x),a)
	\end{align*}
	for all $q\in Q, x\in\Sigma^*, a\in\Sigma$. $\epsilon\in\Sigma^*$ represents the empty string.
\end{defn}
\hfill\\
Hence, $x\in L(M)$ if $\delta^*(q_0,x)\in F$.

\begin{defn}[Regular language]
	A language $L$ is a regular language if $\exists$ some finite automaton $M$ such that $L(M)=L$.
\end{defn}

\end{document}
