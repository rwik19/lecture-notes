\documentclass[10pt, a4paper]{extarticle}
\usepackage{physics}
\usepackage{mathtools}
\usepackage{hyperref}
\usepackage{amssymb}
\usepackage{mathrsfs}
\usepackage[margin=0.75in]{geometry}
\usepackage{anyfontsize}
\usepackage{amsthm}

\theoremstyle{definition}
\newtheorem{thm}{Theorem}
\newtheorem{lem}{Lemma}[thm]
\newtheorem{cor}{Corollary}[thm]
\newtheorem{defn}{Definition}
\newtheorem{eg}{Example}
\newtheorem*{note*}{Note}

\begin{document}
\begin{center}
	\fontsize{25}{60}\selectfont Theory of Computation \\
	\large Based on lectures by Dr. Arpit Sharma\\
	Notes taken by Rwik Dutta
\end{center}
\hrule
\begin{center}
	These notes are not endorsed by the lecturers, and I have modified them (often
	significantly) after lectures. They are nowhere near accurate representations of what
	was actually lectured, and in particular, all errors are almost surely mine.\footnote[1]{This is how Dexter Chua describes his lecture notes from Cambridge. I could not have described mine in any better way.}
\end{center}
\tableofcontents

\newpage

\section{Finite Automata}
\subsection{Deterministic Finite Automaton(DFA)}
\begin{defn}[Deterministic Finite Automaton]
	A collection $(Q,\Sigma,\delta,q_0,F)$ such that
	\begin{enumerate}
		\item $Q$ is a finite set of \emph{states}.
		\item $\Sigma$ is a finite \emph{alphabet}.
		\item $\delta:Q\times\Sigma\to Q$ is the \emph{transition function}.
		\item $q_0\in Q$ is the \emph{start state}.
		\item $F\subseteq Q$ is the set of \emph{accepting states}.
	\end{enumerate}
\end{defn}
We will use automata to solve set membership problems, i.e., given a finite alphabet $\Sigma$ and a finite language $L\subset \Sigma^*$, we need to find whether a given string $x\in L$.

\begin{note*}
	Before a finite automaton has received any input, it is in its initial state, which is an accepting state precisely if the null string is accepted.
\end{note*}

\begin{note*}
	At each step, a finite automaton is in one of a finite number of states (it is a finite automaton because its set of states is finite). Its response depends only on the current state and the current symbol.
\end{note*}

\begin{defn}[String accepted by DFA]
	A DFA has one only one final state for a given string. The string is said to be accepted by the DFA if the final state is also an accepting state of the DFA.
\end{defn}
\subsection{Regular Language}
\begin{defn}[Language of Automaton]
	$L(M)$ of an automaton $M$ is the set of all strings $x$ that are accepted by the automaton.
\end{defn}

\begin{defn}[Extended transition function of DFA]
	Let $M(Q,\Sigma,\delta,q_0,F)$ be a finite automaton. The extended transition function is given by
	\begin{align*}
		\delta^*             & :Q\to\Sigma^*            \\
		\delta^*(q,\epsilon) & =q                       \\
		\delta^*(q,xa)       & =\delta(\delta^*(q,x),a)
	\end{align*}
	for all $q\in Q, x\in\Sigma^*, a\in\Sigma$. $\epsilon\in\Sigma^*$ represents the empty string.
\end{defn}
\hfill\\
Hence, $x\in L(M)$ if $\delta^*(q_0,x)\in F$.

\begin{defn}[Regular language]
	A language $L$ is a regular language if $\exists$ some finite automaton $M$ such that $L(M)=L$.
\end{defn}

\subsection{Set Operations on Regular Languages}
Regular languages are closed under certain operations.

\begin{thm}
	If $M(Q,\Sigma,\delta,q_0,F)$ accepts $L$, $L^C$ is accepted by the finite automaton $M'(Q,\Sigma,\delta,q_0,F')$ where \[F'=Q\backslash F=F^C\]
\end{thm}

\begin{lem}
	$L$ is a regular language $\implies$ $L^C$ is a regular language.
\end{lem}

\begin{thm}
	Let $M_1(Q_1,\Sigma,\delta_1,q_1,F_1)$ accept $L_1$ and $M_2(Q_2,\Sigma,\delta_2,q_2,F_2)$ accept $L_2$. Let $M(Q,\Sigma,\delta,q_0,F)$ be a finite automaton with
	\[Q=Q_1\times Q_2\]\[q_0=(q_1,q_2)\]\[\delta((p,q),a)=(\delta_1(p,a),\delta_2(q,a))\]
	We have $L(M)=$
	\begin{enumerate}
		\item $L_1\cup L_2$ if $F=\{(p,q)|p\in F_1\text{ or }q\in F_2\}$
		\item $L_1\cap L_2$ if $F=\{(p,q)|p\in F_1\text{ and }q\in F_2\}$
		\item $L_1\backslash L_2$ if $F=\{(p,q)|p\in F_1\text{ and }q\notin F_2\}$
	\end{enumerate}
\end{thm}
\begin{lem}
	$L_1,L_2$ are regular languages with the same alphabet $\implies$ $L_1\cup L_2, L_1\cap L_2$ and $L_1\backslash L_2$ are regular languages.
\end{lem}

\begin{thm}[Contatenation]
	If $L_1,L_2$ are regular languages, so is
	\[L_1L_2=\{xy|x\in L_1,y\in L_2\}\]
\end{thm}

\begin{thm}[Kleene Closure]
	If $L$ is a regular language, so is
	\[L^*=\bigcup_{i=0}^{\infty}L^i\]
\end{thm}
\begin{note*}
	$L^0=\{\epsilon\},L^1=L,L^2=LL$ and so on.
\end{note*}
\subsection{Regular Expression}
\begin{defn}[Regular Expression]
	A regular expression over a finite alphabet $\Sigma$ and operators $*,+,\cdot$  is defined as
	\begin{enumerate}
		\item $\varnothing,\epsilon$ and $a\in\Sigma$ are regular expressions.
		\item If $R$ is a regular expression so is $R^*$.
		\item If $R_1,R_2$ are regular expressions so are $R_1+R_2$ and $R_1R_2$.
	\end{enumerate}
\end{defn}
\begin{eg}
	Let $\Sigma=\{0,1\}$.
	$\epsilon,1,0^*,1+0,11,(1+01)^*$ are all examples of regular expressions.
\end{eg}

\begin{defn}[Precedence of Operators]
	$*,\cdot,+$ is the order of decreasing precedence.
\end{defn}

\begin{defn}[Language represented by a regular expression]
	\hfill
	\begin{enumerate}
		\item $\varnothing\rightarrow\varnothing$
		\item $\epsilon\rightarrow\{\epsilon\}$
		\item $a\rightarrow \{a\}$
		\item $R\rightarrow L\implies R^*\rightarrow L^*$
		\item $R_1\rightarrow L_1$ and $R_2\rightarrow L_2\implies R_1+R_2\rightarrow L_1\cup L_2$
		\item $R_1\rightarrow L_1$ and $R_2\rightarrow L_2\implies R_1R_2\rightarrow L_1L_2$
	\end{enumerate}
\end{defn}

\begin{thm}
	Every regular expression represents a unique regular language.
\end{thm}
\begin{note*}
	Every regular language represents a regular expression. However, this expression is not unique.
\end{note*}

\begin{defn}[Equality]
	Two regular expressions are said to be equal if they represent the same language.
\end{defn}

\begin{thm}
	For $\Sigma=\{0,1\}$, $(0^*1^*)^*=(0+1)^*$. Both represent $\Sigma^*$.
\end{thm}

\subsection{Non-deterministic Finite Automaton(NFA)}
\begin{defn}[Non-deterministic Finite Automaton]
	A collection $(Q,\Sigma,\delta,q_0,F)$ such that
	\begin{enumerate}
		\item $Q$ is a finite set of states.
		\item $\Sigma$ is a finite alphabet.
		\item $\delta:Q\times\Sigma_\epsilon\to \mathcal{P}(Q)$ is the transition `function'.
		\item $q_0\in Q$ is the start state.
		\item $F\subseteq Q$ is the set of accepting states.
	\end{enumerate}
\end{defn}
\begin{note*}
	$\Sigma_\epsilon=\Sigma\cup\{\epsilon\}$
\end{note*}
\begin{note*}
	$\mathcal{P}(Q)$ is the set of all subsets of $Q$.
\end{note*}
\begin{note*}
	The transition function of an NFA is not a mathematical function. For a given state $q\in Q$ and symbol $a\in\Sigma$, $\delta(q,a)$ may not exist or may have multiple values in $Q$. This is what mathematicians call a relation.
\end{note*}

\begin{note*}
	The definition of regular languages refers to finite automaton which includes NFA.
\end{note*}

\begin{defn}[String accepted by NFA]
	An NFA can end up in mutiple final states or no state at all with some input string. If \textbf{any one} of these final states is an accepting state of the NFA, we say the NFA accepts the string.
\end{defn}
\end{document}
